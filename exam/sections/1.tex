\subsection{Reguły zapisu instrukcji w Fortranie77}
\begin{itemize}
\item Kolumna 1 : znak C, c lub * oznacza linię komentarza i nie mają
wpływu na wykonanie programu. Komentarze można
umieszczać także po 72 kolumnie lub na prawo od znaku!
\item Kolumny 1-5 : etykieta (ciąg maksymalnie pięciu cyfr, co najmniej
jedna niezerowa; umożliwia odwołanie się do etykietowanej
linii w programie)
\item Kolumna 6 : dowolny znak (różny od zera i spacji) oznacza
kontynuację poprzedniej linii. jedna instrukcja może składać
się maksymalnie z 20. linii (wierszy)
\item Kolumny 7-72 : instrukcje FORTRANu
\end{itemize}

\subsection{Postacie stałej rzeczywistej. Podać przykłady}
\begin{lstlisting}[language=Fortran, caption=dyrektywa implicit]
REAL :: x !12.0, -.3, 1.35E-1
\end{lstlisting}